\renewcommand{\abstractname}{Resumen}

\phantomsection
\addcontentsline{toc}{chapter}{\abstractname}

\thispagestyle{plain}

\renewcommand{\keywords}{Comprensi�n del habla, multiling�ismo, normalizaci�n, segmentaci�n, estrategia \ingles{train-on-target}}

\begin{abstract}
En este proyecto se hace uso de herramientas de traducci�n autom�tica para la creaci�n de un sistema de comprensi�n del habla en ingl�s y franc�s a partir de uno originalmente en castellano. Gracias al uso de estas herramientas se evita la necesidad de una traducci�n y etiquetado manual para la portabilidad a un nuevo idioma. Se ha seguido la aproximaci�n \ingles{train-on-target} para volver a estimar un nuevo modelo de comprensi�n en estos idiomas. Para ello ha sido necesaria la obtenci�n de un conjunto de entrenamiento en ambos idiomas al que se le ha aplicado un proceso de normalizaci�n y otro posterior de segmentaci�n y etiquetado.
\end{abstract}

\renewcommand{\abstractname}{Resum}
\renewcommand{\keywordsname}{Paraules clau}
\renewcommand{\keywords}{Comprensi� de la parla, multiling�isme, normalitzaci�, segmentaci�, estrat�gia \ingles{train-on-target}}

\begin{abstract}
En aquest projecte es fa �s d'eines de traducci� autom�tica per a la creaci� d'un sistema de comprensi� de la parla en angl�s i franc�s a partir d'un obtingut originalment en castell�. Gr�cies a l'�s d'aquestes eines s'evita la necessitat d'una traducci� i etiquetat manual per a la portabilitat a un nou idioma. S'ha seguit l'aproximaci� \ingles{train-on-target} per a tornar a estimar un nou model de comprensi� en aquests idiomes. Per a aix� ha sigut necess�ria l'obtenci� d'un conjunt d'entrenament en ambd�s idiomes al que se li ha aplicat un proc�s de normalitzaci� i altre posterior de segmentaci� i etiquetatge.
\end{abstract}

\renewcommand{\abstractname}{Abstract}
\renewcommand{\keywordsname}{Key words}
\renewcommand{\keywords}{Spoken language understanding, multilingualism, normalization, segmentation, train-on-target strategy}

\begin{abstract}
In this project machine translation tools are used to create a spoken language understanding system in English and French from one originally in Spanish. By using these tools the need of manual translation and labeling is avoided for its portability to a new language. The train-on-target approach has been followed to re-estimate a new understanding model in these languages. In order to accomplish this, a training set in both languages has been required, and a process of normalization and another one of segmentation and labeling have been applied to these sets.

\end{abstract}
